\documentclass[onecolumn, draftclsnofoot,10pt, compsoc]{IEEEtran}
\usepackage{graphicx}
\usepackage{subcaption}
\usepackage{url}
\usepackage{setspace}
%\usepackage{hyperref}
\usepackage{array}

\usepackage{geometry}
\geometry{textheight=9.5in, textwidth=7in}

% 1. Fill in these details
\def \CapstoneTeamName{		100k Challenge CS}
\def \CapstoneTeamNumber{		42}
\def \GroupMemberOne{			Michael Elliott}
\def \GroupMemberTwo{		 	Sam Hudson}
\def \GroupMemberThree{			Glenn Upthagrove}
\def \CapstoneProjectName{		100K Spaceport America Demonstration Rocket Project }
\def \CapstoneSponsorCompany{	School of MIME}
\def \CapstoneSponsorPerson{		Nancy Squires}

% 2. Uncomment the appropriate line below so that the document type works
\def \DocType{		%Technology Review
				%Requirements Document
				%Technology Review
				%Design Document
				Progress Report
				}
			
\newcommand{\NameSigPair}[1]{\par
\makebox[2.75in][r]{#1} \hfil 	\makebox[3.25in]{\makebox[2.25in]{\hrulefill} \hfill		\makebox[.75in]{\hrulefill}}
\par\vspace{-12pt} \textit{\tiny\noindent
\makebox[2.75in]{} \hfil		\makebox[3.25in]{\makebox[2.25in][r]{Signature} \hfill	\makebox[.75in][r]{Date}}}}
% 3. If the document is not to be signed, uncomment the RENEWcommand below
\renewcommand{\NameSigPair}[1]{#1}

%%%%%%%%%%%%%%%%%%%%%%%%%%%%%%%%%%%%%%%
\begin{document}
\begin{titlepage}
    \pagenumbering{gobble}
    \begin{singlespace}
    	\includegraphics[height=4cm]{coe_v_spot1} %HERE
        \hfill 
        % 4. If you have a logo, use this includegraphics command to put it on the coversheet.
        %\includegraphics[height=4cm]{CompanyLogo}   
        \par\vspace{.2in}
        \centering
        \scshape{
            \huge CS Capstone \DocType \par
            {\large\today}\par
            \vspace{.5in}
            \textbf{\Huge\CapstoneProjectName}\par
            \vfill
            {\large Prepared for}\par
            \Huge \CapstoneSponsorCompany\par
            \vspace{5pt}
            {\Large\NameSigPair{\CapstoneSponsorPerson}\par}
            {\large Prepared by }\par
            Group\CapstoneTeamNumber\par
            % 5. comment out the line below this one if you do not wish to name your team
            %\CapstoneTeamName\par 
            \vspace{5pt}
            {\Large
                \NameSigPair{\GroupMemberOne}\par
                \NameSigPair{\GroupMemberTwo}\par
                \NameSigPair{\GroupMemberThree}\par
            }
            \vspace{20pt}
        }
        \begin{abstract}
        % 6. Fill in your abstract    
	The software developed for the 100k Spaceport America challenge shall collect telemetry data, filter noise, transmit to a ground station, create a visualization, and log the data. The data shall travel down a pipeline starting on the rocket and ending at the visualization on the display. The following document details the progress made on this progress for the Fall term of 2017. 
        \end{abstract}     
    \end{singlespace}
\end{titlepage}
\newpage
\pagenumbering{arabic}
\tableofcontents
% 7. uncomment this (if applicable). Consider adding a page break.
%\listoffigures
%\listoftables
\clearpage

% 8. now you write!
\section {Purpose and Goals}
The Spaceport America 100k rocketry challenge is a contest in which the Oregon State University chapter of the American Institute of Aeronautics and Astronautics (AIAA) is competing. The rocket is being designed and created by an interdisciplinary team of engineering seniors, from both the School of Mechanical Industrial and Manufacturing Engineering and the School of Electrical Engineering and Computer Science. The 18 person team contains engineering students from various disciplines, combining their skills to create, launch and track a high altitude rocket. The goal is for the rocket to exceed 100,000 feet in altitude. The specific goal of the computer science sub-team is to develop software to effectively track the rocket, successfully recover it, and to visualize its flight path. \par
\section {Current Progress} 
The first term of this year was spent on research and design. We have identified what our software must do, the appropriate tools to utilize, and we have a solid strategy in place for how to make the system. This time was far from wasted, as design is just as important as implementation, and taking the time to understand the problems more fully will lead to a better product. This does mean however that we have little to no actual code as of this moment. There have been proof of concepts done for the 3D trace and the 2D visualization, but nothing concrete has been written. \\
While having concrete code would be a good milestone to have reached this term, it is not a problem. The system is designed, a testing plan in place, and work evenly divided. Over the course of the winter holiday and the winter term of the school year the entire system can be created and tested, thanks to the time spent on adequate design. \par
\section {Retrospective}
\begin {center}
 \begin {tabular} { | p{5cm} | p{5cm} | p{5cm} | }
 \hline
 Positives & Deltas & Actions \\
 \hline
 The AIAA general meetings and the sub-team meetings every week gave us two extra chances beyond the classroom to meet and discuss what was needed both within our group and in relation to the other sub-teams. & While the meetings were helpful, they would be far more effective if all group members were there for the majority of the meetings. The group work we do should also begin at earlier times, as some of our projects were completed far too close to the time they were due. & Better communication and planning can be achieved by using a project management tool. We can better plan out our work and then have access to those plans at any time. There should also be a set number of meetings you can miss per term, such as two, and you have to inform at least one other group member at least an hour in advance. \\
 \hline
 \end {tabular}
\end {center} 
\section {Detailed Development}
\subsection {Week 1}
\subsubsection{Michael Elliott}
\begin {itemize}
\item \textbf{Plans: }
 This week I plan to get a team together to work on the 100K Spaceport America Demonstration Rocket Project. This is the project that interests me the most and I have already talked to the members of the other teams. The next steps include finding 2 more members for the CS team and talking to Nancy Squires, the project's client, about requesting us as her team.
\item \textbf{Problems: }
  The biggest problem I ran in to during this first week was finding other talented individuals who wanted to work on the project. Many showed interest but did not want to make a commitment.
\item \textbf{Progress: }
  I have found 2 other members for our team and have been in contact with Dr. Squires. Hopefully we will be able to solidify our positions on the team. We also attended the first all team meeting this week since work on the project needs to be started immediately due to the reduced timeline even though we have not been confirmed for the project yet. We used this meeting to introduce ourselves to each other and familiarize ourselves with the project. We also held a second team meeting with all of the avionics teams to start very high level planning.
\end {itemize}
\subsubsection{Sam Hudson}
\begin {itemize}
\item \textbf{Plans: }Discover what is required for CS 461.
\item \textbf{Problems: }Some issues establishing best meeting times for group.
\item \textbf{Progress: }We determined that Monday was the best time to meet with all groups related to the project and Wednesday was the best time to meet for our subteam meeting.
\item \textbf{Summary: }This week I met with the 100k rocket team in Rogers. We discussed project management strategies and I suggested to use a project management tool called Jira. I configured Jira for the team to try. We discussed a few resources to consider for the project including a book discussing aspects of test driven development for embedded C. Towards the end of the meeting I submitted my project preferences. 
\end {itemize}
\subsubsection{Glenn Upthagrove}
\begin {itemize}
 \item \textbf{Plans: }Find projects that interest me. 
 \item \textbf{Problems: }We are having trouble getting in touch with the professor in charge of the AIAA 100k challenge team. 
 \item \textbf{Progress: }I have discovered an opportunity to be on the 100k rocket challenge avionics team and I have been to the first two meetings. It looks very interesting and challenging.  
 \item \textbf{Summary: }This week I started out uncertain as to which projects I would be interested in. I searched through them for a day or two and then was presented with the chance to visit the first meeting of the AIAA 100k challenge. I have been taken onto the team for almost certain. I am excited for this challenge and I look forward to it as we have already started to collaborate, brainstorm and review last years work to develop a plan for improvements and revisions.  
\end {itemize}
\subsection {Week 2}
\subsubsection{Michael Elliott}
\begin {itemize}
\item \textbf{Plans: }
  Go to the AIAA meeting Wednesday at 6pm in LPSC 125 as well as the OROC mentors meeting Saturday at 11am in Rogers. Start defining the problems we will need to solve and identifying potential solutions.
\item \textbf{Problems: }
  We didn't have any serious issues this week.
\item \textbf{Progress: }
  I met with Keith Packard from OROC, the creator of the Telemega. We discussed the capabilities of the Telemega as well as some of the problems we might encounter. We also discussed some of the problems encountered by last year's team. I talked to him about Kalman filters and he explained how they work. Additionally, I did my own research into the capabilities of the Telemega and Kalman filters. I also went to the AIAA meeting with Glenn to talk about our team to other students. We met with Dr. Squires in person as well after the meeting.
\end {itemize}
\subsubsection{Sam Hudson}
\begin {itemize}
\item \textbf{Plans: }Understand what is involved with the 100k Rocketry Challenge.
\item \textbf{Problems: }Getting to terms with what is expected from a computer science perspective. 
\item \textbf{Progress: }Received an overview of the requirement specification from the professor leading the project and other sub teams involved.
\item \textbf{Summary: }This week went over the structure of the course and determined what was the best method for approaching assignments. Very useful feedback was given.
\end {itemize}
\subsubsection{Glenn Upthagrove}
\begin {itemize}
 \item \textbf{Plans: }Go to AIAA meeting and introduce ourselves to client
 \item \textbf{Problems: }We are having trouble scheduling a time we can all meet.
 \item \textbf{Progress: }We have met the client in person and sent an introduction email. We still need to work on scheduling for all three of us. 
 \item \textbf{Summary: }This week we focused on contacting the client and the other capstones groups we needed to contact due to this being an interdisciplinary project. The client offered us the option to adjust requirements if necessary later in the year, to accommodate the fact that the other teams have deliverables in the winter. We attended an AIAA meeting and presented our team's mission to other undergraduates in the hopes that they may get involved. 
\end {itemize}
\subsection {Week 3}
\subsubsection{Michael Elliott}
\begin {itemize}
\item \textbf{Plans: }
  We plan to do more research into the problems and solutions discussed with Keith and work on the Problem Statement. Do research on transmission protocols and the types and causes of sensor noise.
\item \textbf{Problems: }
  While we didn't have much trouble identifying problems, we did have a bit of trouble identifying solutions to those problems and even more so we had trouble coming up with ways to gauge our success.
\item \textbf{Progress: }
  We worked on identifiying the problems we will need to solve and writing our Problem Statement assignment. I did research into possible solutions to our problems. We created our goals for the term and made sure they were attainable.
\end {itemize}
\subsubsection{Sam Hudson}
\begin {itemize}
\item \textbf{Plans: }Determine what’s required for the problem statement assignment.
\item \textbf{Problems: }Effectively categorizing a list of goals for the team to work towards throughout the year.
\item \textbf{Progress: }Discovered what contributions could be made and some proposed solutions.
\item \textbf{Summary: }This week we gathered a list of all contributions to the project from an avionics perspective. We discussed with the ECE team how things might integrate.
\end {itemize}
\subsubsection{Glenn Upthagrove}
\begin {itemize}
 \item \textbf{Plans: }Get in contact with client to speak about requirements. Write problem statement rough draft. Order book. Set up GitHub
 \item \textbf{Problems: }We had some trouble with understanding what our performance metrics could and should be. 
 \item \textbf{Progress: }We have met the client in person and were given some ideas. I spoke with McGrath about this as well. I have written the rough draft of my problem statement. I have set up GitHub. 
 \item \textbf{Summary: }This week we focused on the problem statements. We have rough drafts for each of us and The GitHub is now live. We invited both professor McGrath and professor Winters as collaborators. There is still some polish to be done on my rough draft of the problem statement. The expectations of us are still very vague at this point but professor McGrath expects this and I am doing my best with my team to come up with reasonable promises, as well as some things we would like to do. 
\end {itemize}
\subsection {Week 4}
\subsubsection{Michael Elliott}
\begin {itemize}
\item \textbf{Plans: }
  This week we need to figure out our budget, coordinate our goals with the goals of the ECE and ME avionics teams, and write a mission statement for the project. We also need to begin solidifying high level designs for how our programs are going to fit together. In addition we need to finish our final draft of our problem statement.
\item \textbf{Problems: }
  We had a small hiccup combining our individual problem statements into one cohesive group one. Git helped us out.
\item \textbf{Progress: }
  I created and submitted our budget proposal. We met with the ECE team to discuss our possible solutions we came up with for the Problem Statement. We also finished the final draft of the Problem Statement and submitted it.
\end {itemize}
\subsubsection{Sam Hudson}
\begin {itemize}
\item \textbf{Plans: }Re-determine what the best method is for tracking project.
\item \textbf{Problems: }Had to reevaluate the best tool to use for tracking because of budget constraints.
\item \textbf{Progress: }I looked into different tools to track issues throughout the project going forward. I initially had considered using Jira as a tool for project management. Although this tool wasn't as familiar to everyone on the team so we made an active decision to use Github issue tracking. 
\item \textbf{Summary: }This week I did some research surrounding unidirectional protocols to get a better idea of what some limitations could be. Some materials on the rocket can affect the capability receiving signals. 
\end {itemize}
\subsubsection{Glenn Upthagrove}
\begin {itemize}
 \item \textbf{Plans: }Finish the problem statement. We need to make a forecast for expenses. We also need an outreach statement. 
 \item \textbf{Problems: }We are not certain how to make this a 10,000 foot view of the problem. We are getting a bit too technical in our description. 
 \item \textbf{Progress: }We have received feedback and have edited some parts of the description. We decided we need no money. We also wrote an outreach statement. 
 \item \textbf{Summary: }This week we focused on writing the problem statement and the outreach statement for AIAA. I have also personally looked more into Kalman filters and I have found some books that I may want to buy or check out at the library.  
\end {itemize}
\subsection {Week 5}
\subsubsection{Michael Elliott}
\begin {itemize}
\item \textbf{Plans: }
  This week we will set up meeting with our TA and start working on our requirements.
\item \textbf{Problems: }
  I was sick so I didn't meet all my personal goals for the week. Fortunately this will not impact us negatively in the long run.
\item \textbf{Progress: }
  I made contact at JPL who worked on the software for the Cassini project as well as others. He had some really good insights into how to go about solving certain problems. We also met with our TA and set up a weekly meeting time.
\item \textbf{Summary: }
\end {itemize}
\subsubsection{Sam Hudson}
\begin {itemize}
\item \textbf{Plans: }Meet with T/A to establish what level of help we may need.
\item \textbf{Problems: }Determining a time that best suits all parties for meeting with T/A.
\item \textbf{Progress: }Determined the best time to meet during the week remotely on Tuesdays at 3:30pm.
\item \textbf{Summary: }This week, we established the best meeting time, worked on Github naming conventions and gave some good feedback on writing posts.
\end {itemize}
\subsubsection{Glenn Upthagrove}
\begin {itemize}
 \item \textbf{Plans: }Make a rough draft of the requirements document.
 \item \textbf{Problems: }Two of our team members are sick, so progress on all fronts has been slowed.
 \item \textbf{Progress: }The rough draft of the requirements document is done and an appointment is set up with the client for Monday to double check our understanding. 
 \item \textbf{Summary: }This week we focused on the requirements document. We also met with our TA for the first time. We have set up an appointment with Dr. Squires to make sure that we understand her expectations before the final draft is done.  
\end {itemize}
\subsection {Week 6}
\subsubsection{Michael Elliott}
\begin {itemize}
\item \textbf{Plans: }
  This week we need to finalize some decisions in order to start making more progress with our project. We also need to finish and submit our requirements document.
\item \textbf{Problems: }
  We had some trouble reaching out to the client about our requirements document.
\item \textbf{Progress: }
  We finished our requirements document and started making design decisions and researching the technical implications of various options. We also met with the ECE team to coordinate their hardware with our solution proposals.
\end {itemize}
\subsubsection{Sam Hudson}
\begin {itemize}
\item \textbf{Plans: }To discuss package sizes and package composition.
\item \textbf{Problems: }Understand exactly what’s required for the requirements document.
\item \textbf{Progress: }We meet 3 times this week to discuss technologies that might be required to provide our solution.
\item \textbf{Summary: }This week I researched visualization technologies and also I worked on the requirements document and fixed some of the formatting.
\end {itemize}
\subsubsection{Glenn Upthagrove}
\begin {itemize}
 \item \textbf{Plans: }Meet with client and make final draft. 
 \item \textbf{Problems: }We are having trouble getting in contact with the client.
 \item \textbf{Progress: }We have a document and we have emailed the professors and the client, but we may or may not need to make changes after the due date if she disagrees with the current draft.
 \item \textbf{Summary: }This week we focused on the requirements document. We are trying to get in contact with our client, but being as busy as she is it is understandable that she has not gotten in contact with us. We are going to keep trying to contact the client and we will submit what we have. Hopefully, seeing as I have talked with professor Winters and emailed both her and professor McGrath, if changes are needed we can easily make them.  
\end {itemize}
\subsection {Week 7}
\subsubsection{Michael Elliott}
\begin {itemize}
\item \textbf{Plans: }
  This week we will compare and contrast different options for potential design decisions. We also plan to finalize these decisions and discuss the decisions with the ECE team.
\item \textbf{Problems: }
  I've been having some issues with finding convincing alternatives for technologies that are already clearly the best choice for the application. My conversation with my contact at JPL certainly helped with this.
\item \textbf{Progress: }
  I began preliminary packet design for the transmission protocol. I also explored alternative programming languages, static analyzers, protocol models, and raw data interpretation methods.
\end {itemize}
\subsubsection{Sam Hudson}
\begin {itemize}
\item \textbf{Plans: }Look into different web technologies for frontend and backend web application development.
\item \textbf{Problems: }There was some discussion around whether there would be two onboard systems for monitoring telemetry at different stages of the launch.
\item \textbf{Progress: } The resolution was to have two onboard systems for telemetry acquisition. I did some research around these different web technologies such as Flask, MongoDB and MaterializeCSS.
\item \textbf{Summary: }This week, we had a lot of discussion around where the onboard systems will be placed and how materials and placement would effect telemetry instruments. In addition found some great resources for front-end development.
\end {itemize}
\subsubsection{Glenn Upthagrove}
\begin {itemize}
 \item \textbf{Plans: }Figure out our individual modules for research and start technology review.
 \item \textbf{Problems: }I am having trouble finding an appropriate third module for me. 
 \item \textbf{Progress: }We have made a list of modules that we can each research, as well as technologies we can start looking into. 
 \item \textbf{Summary: }This week we focused on the technology review. We have met together and figured out modules for each of us to research. We also have some ideas for the technologies to look into. We also spoke with the electrical team about packet length. 
\end {itemize}
\subsection {Week 8}
\subsubsection{Michael Elliott}
\begin {itemize}
\item \textbf{Plans: }
  This week we will be working on our Tech Reviews and finalizing all of our design decisions. We will be working closely with the ECE subteam in communicating our design decisions and our needs as well as making sure our decisions don't get in the way of their needs.
\item \textbf{Problems: }
  I forgot to bring my Tech Review draft to class on the day we did peer reviews.
\item \textbf{Progress: }
  We finalized the major design decisions that we needed to make in order to begin making more detailed plans of execution. We prepared ourselves for the coming short week.
\end {itemize}
\subsubsection{Sam Hudson}
\begin {itemize}
\item \textbf{Plans: }To establish responsibility of contribution for each team member.
\item \textbf{Problems: }I was sick towards the end of the week which limited my productivity. 
\item \textbf{Progress: }During the beginning of the week I was able to convey some research I conducted to the team.
\item \textbf{Summary: }This week, I presented to the rocketry group discussing different types of monitoring tools and discussed the web backend that I will standup for the team. We discussed regression and different statistical techniques to create estimates of trajectory etc.
\end {itemize}
\subsubsection{Glenn Upthagrove}
\begin {itemize}
 \item \textbf{Plans: }Discuss the tech review more. Get more solid ideas on the ground station setup. 
 \item \textbf{Problems: }I am personally trying to find a time to speak with the client, but it is proving difficult. 
 \item \textbf{Progress: }I have found office hours posted and I will also try to email the client to try and set up a meeting.
 \item \textbf{Summary: }This week I was considering more deeply how the ground station should be set up and how my code will run on it. I also have confirmed that Dr. Yavuz will let me discuss some networking topics with him. Dr. Bailey has confirmed I am allowed to use any code he provided during CS450 in my code for the 3D trace.  
\end {itemize}
\subsection {Week 9}
\subsubsection{Michael Elliott}
\begin {itemize}
\item \textbf{Plans: }
  There is not much to do this week due to Thanksgiving break besides the work for the class. We still need to meet with the other subteams and our TA to coordinate.
\item \textbf{Problems: }
  Thanksgiving break is making it hard to coordinate schedules and get work done.
\item \textbf{Progress: }
  I completed my Tech Review and worked on my Design Document. We met with the other subteams and our TA to talk about our progress.
\end {itemize}
\subsubsection{Sam Hudson}
\begin {itemize}
\item \textbf{Plans: }More research around frameworks and data transactions. 
\item \textbf{Problems: }A lot of thought into determining the best frameworks to use.
\item \textbf{Progress: }The tech review helped narrow down which would be the best options to use.
\item \textbf{Summary: }This week, we had a discussion around different frameworks that would be used in the construction of the web application to organize the telemetry data.  The frameworks were broken down into three different sections. Chart/Graphic generation frameworks, front-end frameworks and server-side frameworks. This was used as a basis for the tech review document. 
\end {itemize}
\subsubsection{Glenn Upthagrove}
\begin {itemize}
 \item \textbf{Plans: }Finish the technology review. 
 \item \textbf{Problems: }I am personally not certain that much work shall be done this week as this thursday thanksgiving. I must finish the technology review this week.
 \item \textbf{Progress: }I have finished the technology review and set a number of upcoming goals for the actual programming coming up. 
 \item \textbf{Summary: }This week I finished the technoloy review. I am not doing much else this week due to the holiday, and also personal medical concerns. I will be working over the break, but also spending time with doctors, so progress may be minimal. 
\end {itemize}
\subsection {Week 10}
\subsubsection{Michael Elliott}
\begin {itemize}
\item \textbf{Plans: }
  For this week we need to finish the Design Document and meet with the ECE subteam to coordinate our winter break schedules and create preliminary deadlines.
\item \textbf{Problems: }
  Unfortunately I was unable to attend the all team meeting this week. I was still able to brief the other CS team members on what needed to be discussed as well as obtain notes from the meeting.
\item \textbf{Progress: }
  
\end {itemize}
\subsubsection{Sam Hudson}
\begin {itemize}
\item \textbf{Plans: }Design document and make a final decision surrounding technologies.
\item \textbf{Problems: }Finding the best time to meet as we all had a busy week with final assignments.
\item \textbf{Progress: }Together we finalized the design document.
\item \textbf{Summary: }This week we made a final decision on the frameworks we are going to use. We decided on D3.JS to handle graphics, materializeCSS for use on the frontend and flash with MongoDB as our server-side framework. There was an Impromptu style Q\&A where I went to learn more about how the design document should be structured. 
\end {itemize}
\subsubsection{Glenn Upthagrove}
\begin {itemize}
 \item \textbf{Plans: }Make the final draft of the design document. 
 \item \textbf{Problems: }I was under the impression the rough draft was due this Friday, but it was actually last Friday, and thus I am concerned. Running OpenGL code on the RaspberryPi I have borrowed is proving more difficult than I had imagined.
 \item \textbf{Progress: }I know Dr. Winters has office hours on Wednesday, and I can speak with her then.
 \item \textbf{Summary: }This week I am focusing on the design document. I also have learned how to set up continuous integration testing on our GitHub.
\end {itemize}


%\bibliographystyle{IEEEtran}
%\bibliography{bibliography}


\end{document}
