\documentclass[onecolumn, draftclsnofoot,10pt, compsoc]{IEEEtran}
\usepackage{graphicx}
\usepackage{url}
\usepackage{setspace}

\usepackage{geometry}
\geometry{textheight=9.5in, textwidth=7in}

\def \CapstoneTeamName{		100k Challenge CS}
\def \CapstoneTeamNumber{		42}
\def \GroupMemberOne{		 	Sam Hudson}

\def \CapstoneProjectName{		100K Spaceport America Demonstration Rocket Project}
\def \CapstoneSponsorCompany{	School of Mechanical Engineering, Oregon State University}
\def \CapstoneSponsorPerson{		Nancy Squires}

% 2. Uncomment the appropriate line below so that the document type works
\def \DocType{  %Problem Statement
				%Requirements Document
				%Technology Review
				%Design Document
				%Progress Report
				   Final Report
				}
			
\newcommand{\NameSigPair}[1]{\par
\makebox[2.75in][r]{#1} \hfil 	\makebox[3.25in]{\makebox[2.25in]{\hrulefill} \hfill		\makebox[.75in]{\hrulefill}}
\par\vspace{-12pt} \textit{\tiny\noindent
\makebox[2.75in]{} \hfil		\makebox[3.25in]{\makebox[2.25in][r]{Signature} \hfill	\makebox[.75in][r]{Date}}}}
% 3. If the document is not to be signed, uncomment the RENEWcommand below
%\renewcommand{\NameSigPair}[1]{#1}

%%%%%%%%%%%%%%%%%%%%%%%%%%%%%%%%%%%%%%%
\begin{document}
\begin{titlepage}
    \pagenumbering{gobble}
    \begin{singlespace}
    	\includegraphics[height=4cm]{coe_v_spot1}
        \hfill 
        % 4. If you have a logo, use this includegraphics command to put it on the coversheet.
        %\includegraphics[height=4cm]{CompanyLogo}   
        \par\vspace{.2in}
        \centering
        \scshape{
            \huge CS Capstone \DocType \par
            {\large\today}\par
            \vspace{.5in}
            \textbf{\Huge\CapstoneProjectName}\par
            \vfill
            {\large Prepared for}\par
            \Huge \CapstoneSponsorCompany\par
            \vspace{5pt}
            {\Large\NameSigPair{\CapstoneSponsorPerson}\par}
            {\large Prepared by }\par
            Group\CapstoneTeamNumber\par
            % 5. comment out the line below this one if you do not wish to name your team
            \CapstoneTeamName\par 
            \vspace{5pt}
            {\Large
                \NameSigPair{\GroupMemberOne}\par
            }
            \vspace{20pt}
        }
        \begin{abstract}
        % 6. Fill in your abstract    
        This document will review technologies that will be considered for use in the design and execution of our final product. 
	\end{abstract}     
    \end{singlespace}
\end{titlepage}
\newpage
\pagenumbering{arabic}
\tableofcontents
% 7. uncomment this (if applicable). Consider adding a page break.
%\listoffigures
%\listoftables
\clearpage

\section{Objectives}
\subsection{Goals}
\subsection{Purposes}


\section{Progress}
During the deployment, in-flight and descent stages we will be tracking the rocket through radio telemetry. This data will be sent from a transmitter attached to the rocket down to the ground station. When the data is received by the ground station the data will then need to be organized and displayed to all parties following the launch. We will be designing a web application that imports telemetry data and renders useful information graphically such as trajectory, altitude and velocity. There are many frontend libraries for representing data graphically. Listed below are three contenders for providing this functionality. 


\section{Retrospective}
\subsection{Week 1}
This week I met with team to touch base on progress over the winter break. Discussed a few different a strategies to approach the development and worked with ECE team to understand exactly what was required of us to support them. 
\subsection{Week 2}
This week I configured a Docker environment which included a database container, API container and Web Application container. I created a docker-compose file which built images with each containers requirements. For example the API container has Flask installed with the requests library and the database container has mongo db installed.
\subsection{Week 3}
This week was a productive week I managed to implement the API endpoints for posting and getting data from mongodb. I also managed create a basic web application which made requests to the database via the API. This implementation was purposefully designed to be technology agnostic.  
\subsection{Week 4}
This week I helped Glenn implement a data handler to interface with the API and handles data generated from the data generator. Once this was developed we were able to simulate some telemetry transactions and see the update in real time on the web application. I implemented an ASYNC function in javascript that polled the database via the API to ensure new information was updated to the web application once per second.  
\subsection{Week 5}
This week I worked on firmware with ECE. Glenn and I talked worked through the logic for interfacing with the sensors. I wrote the SPI read and write functions in C for passing data. We are just waiting ECE to flash the board so we can install the firmware to actually test this. This week I also managed to create some D3 visualizations. Have not integrated this get with the ASYNC function. 
\subsection{Week 6}
This week I worked some more on the firmware functions I had taken a look into the timer component we will be using two timers. One for firing the FET and one for tracking the duration of the flight. We haven't had the ability to test this code on the hardware. We are currently waiting for access to the PCB.
\subsection{Week 7}
This week my focus was on integrating the gauges for speed, velocity and altitude with the ASYNC JavaScript function. I found some D3 gauges that worked well with the ascetics of the page. I ended up writing a gauges module that consisted of an initialization function, an update function and some unit conversion functions.
\subsection{Week 8}
This week my focus was to research into different chart libraries that I could use for plotting altitude against time. I found a library called C3 that I discovered initially in my research. I then used this library to develop a time series graph that plots altitude against time.
\subsection{Week 9}
This week my focus was on architecting the API, web application and database to support launch data and both sustainer and booster types of telemetry. This was an important modification because it allowed us to show how these two types of telemetry distinctly on the web application. Now altitude data points for both the sustainer and the booster are plotted on the time series chart and a gauge duo for each telemetry is now available on the web application.
\subsection{Week 10}
My focus this week has been mainly geared towards firmware modifications and making final modifications to the web application. Two stretch goals that I have set for myself for next term is to add a live feed of the camera on the rocket and to show staging throughout flight. I want to implement a progress bar that shows stages accurately based on telemetry data received. 

%~250
%the special role you believe each member, including yourself, has played on the team (e.g., manager, technical expert, etc.),
%the level of contribution (e.g., higher/lower than, equal to, different from) for each member, including yourself, and
%how well you believe your group functions as a development team

\section{Team}
\subsection{Roles}

Samuel Hudson 
I'm responsible for developing the web tools designed to track the rocket throughout it's flight. These tools include the API, web application and database. My job is to ensure that telemetry data transmitted from the rocket syncs correctly with the web application and is visualized correctly. I believe I was best suited to this role because of my background in web development. In the past I have worked on many different web projects that involve the creation of RESTful API backends. I have a strong understanding of python and JavaScript. In addition to web tools development I have been actively involved in the firmware portion of the project. I have an understanding of how most functions interface with the hardware.

Glenn Upthegrove

Glenn has a strong knowledge of C/C++ and assembly languages. Glenn is mainly responsible for writing and maintaining the ground station software for receiving and converting telemetry data from the rocket into JSON (JavaScript Object Notation). In addition Glenn and I created a data handler to push JSON telemetry data to the MongoDB storing the telemetry data for the API to interface with. Other than the development of the ground station software Glenn has been actively involved in firmware development. Glenn has written the main component for the 


\subsection{Level of Contribution}

Samuel Hudson

I have been mainly contributing to the web tools portion of the project. For the web application portion of the project I have been programming mainly in Python 
\subsection{Coordination}

\section{Conclusion}

\begin{thebibliography}{9}
\bibitem{1} 
S. F. Rahman, “The 15 Best JavaScript Charting Libraries,” SitePoint, 12-Sep-2017. [Online]. Available: https://www.sitepoint.com/15-best-javascript-charting-libraries/. [Accessed: 21-Nov-2017].

\bibitem{2}
“D3.js or Chartjs? Which to use when,” Navyug, 09-Nov-2017. [Online]. Available: http://navyuginfo.com/d3-js-chartjs-use/. [Accessed: 21-Nov-2017].

\bibitem{3}
d3, “d3/d3,” GitHub. [Online]. Available: https://github.com/d3/d3/wiki. [Accessed: 21-Nov-2017].

\bibitem{4}
Chartjs, “Chart.js,” Chart.js · GitBook. [Online]. Available: http://www.chartjs.org/docs/latest/. [Accessed: 21-Nov-2017].


\bibitem{5}
“c3 VS Chart.js,” c3 vs Chart.js | LibHunt. [Online]. Available: https://js.libhunt.com/project/c3/vs/chart-js. [Accessed: 21-Nov-2017].

\bibitem{6}
c3js.org, “C3.js | D3-based reusable chart library,” C3.js | D3-based reusable chart library. [Online]. Available: http://c3js.org/. [Accessed: 21-Nov-2017].

\bibitem{7}
“C3.js Brings Charting Power Without the Learning Curve,” InfoQ. [Online]. Available: https://www.infoq.com/news/2014/09/c3js-d3-charting. [Accessed: 21-Nov-2017].

\bibitem{8}
“Materialize vs. Bootstrap? Which is Best Front-End Framework?,” Stunning Mesh, 22-Jun-2017. [Online]. Available: http://www.stunningmesh.com/2016/11/materialize-vs-bootstrap-best-front-end-framework/. [Accessed: 21-Nov-2017].

\bibitem{9}
“Materialize,” Documentation - Materialize. [Online]. Available: http://materializecss.com/. [Accessed: 21-Nov-2017].

\bibitem{10}
W. by H. Chouhan, “6 Reasons to Choose the Bootstrap CSS Framework,” OSTraining. [Online]. Available: https://www.ostraining.com/blog/coding/bootstrap/. [Accessed: 21-Nov-2017].

\bibitem{11}
“Bootstrap vs Semantic UI vs Materialize 2017 Comparison,” StackShare. [Online]. Available: https://stackshare.io/stackups/bootstrap-vs-materialize-vs-semantic-ui. [Accessed: 21-Nov-2017].

\bibitem{12}
G. Dwyer, “Flask vs. Django: Why Flask Might Be Better,” Codementor. [Online]. Available: https://www.codementor.io/garethdwyer/flask-vs-django-why-flask-might-be-better-4xs7mdf8v. [Accessed: 21-Nov-2017].
\bibitem{13}
“Express Web Framework (Node.js/JavaScript),” Mozilla Developer Network. [Online]. Available: https://developer.mozilla.org/en-US/docs/Learn/Server-side/Express\_Nodejs. [Accessed: 21-Nov-2017].



\end{thebibliography}






\end{document}

