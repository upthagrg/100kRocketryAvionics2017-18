\documentclass[onecolumn, draftclsnofoot,10pt, compsoc]{IEEEtran}
\usepackage{graphicx}
\usepackage{url}
\usepackage{setspace}

\usepackage{geometry}
\geometry{textheight=9.5in, textwidth=7in}

\def \CapstoneTeamName{		OSU HART}
\def \CapstoneTeamNumber{		42}
\def \GroupMemberOne{			Michael Elliott}
\def \GroupMemberTwo{			Glenn Upthagrove}
\def \GroupMemberThree{			Sam Hudson}
\def \CapstoneProjectName{		100K Spaceport America Demonstration Rocket Project}
\def \CapstoneSponsorCompany{	Mechanical Engineering, Oregon State University}
\def \CapstoneSponsorPerson{		Nancy Squires}

% 2. Uncomment the appropriate line below so that the document type works
\def \DocType{  Problem Statement
				%Requirements Document
				%Technology Review
				%Design Document
				%Progress Report
				}
			
\newcommand{\NameSigPair}[1]{\par
\makebox[2.75in][r]{#1} \hfil 	\makebox[3.25in]{\makebox[2.25in]{\hrulefill} \hfill		\makebox[.75in]{\hrulefill}}
\par\vspace{-12pt} \textit{\tiny\noindent
\makebox[2.75in]{} \hfil		\makebox[3.25in]{\makebox[2.25in][r]{Signature} \hfill	\makebox[.75in][r]{Date}}}}
% 3. If the document is not to be signed, uncomment the RENEWcommand below
%\renewcommand{\NameSigPair}[1]{#1}

%%%%%%%%%%%%%%%%%%%%%%%%%%%%%%%%%%%%%%%
\begin{document}
\begin{titlepage}
    \pagenumbering{gobble}
    \begin{singlespace}
    	\includegraphics[height=4cm]{coe_v_spot1}
        \hfill 
        % 4. If you have a logo, use this includegraphics command to put it on the coversheet.
        %\includegraphics[height=4cm]{CompanyLogo}   
        \par\vspace{.2in}
        \centering
        \scshape{
            \huge CS Capstone \DocType \par
            {\large\today}\par
            \vspace{.5in}
            \textbf{\Huge\CapstoneProjectName}\par
            \vfill
            {\large Prepared for}\par
            \Huge \CapstoneSponsorCompany\par
            \vspace{5pt}
            {\Large\NameSigPair{\CapstoneSponsorPerson}\par}
            {\large Prepared by }\par
            Group\CapstoneTeamNumber\par
            % 5. comment out the line below this one if you do not wish to name your team
            \CapstoneTeamName\par 
            \vspace{5pt}
            {\Large
                \NameSigPair{\GroupMemberOne}\par
                \NameSigPair{\GroupMemberTwo}\par
                \NameSigPair{\GroupMemberThree}\par
            }
            \vspace{20pt}
        }
        \begin{abstract}
          In this project we will design and build recovery and telemetry systems for a rocket that will reach a target altitude of 100,000 feet.
          We will create telemetry interfaces for tramsmitting and receiving data and avionics for the recovery of the rocket.
          This problem statement document will cover the details of the challenge, the outline of the proposed solution, and metrics to judge the success of the project.
          The main challenge will be to design a means of collecting, interpreting, and transmitting data on the rocket, in addition to receiving, interpreting, and displaying data on the ground.
          Additionally, we will be working with embedded hardware, the constraints of which we will have to work within.
          Our project will be successful if we can collect, transmit, and properly utilize the flight data, both in flight and to recover the rocket.
        \end{abstract}     
    \end{singlespace}
\end{titlepage}
%\newpage
\pagenumbering{arabic}
% 7. uncomment this (if applicable). Consider adding a page break.
%\tableofcontents
%\listoffigures
%\listoftables
\clearpage

\section{Problem Definition}
In sending a rocket 100,000 feet in the air, there are a few challenges that must be met.
The first challenge is collecting accurate data.
At speeds above Mach 1 and in thinner atmosphere, the data from the sensors will not be as accurate.
We will have to find a way to collect and utilize the data we receive from the sensors even though it may be inaccurate.
We will also have to ensure that the altitude reading is both accurate and available on the ground.
This leads to our next big challenge: transmission of data.
We will have to create our own protocol to transmit and receive data while travelling at high speeds in thin atmosphere, possibly with RF interference.
We will have to ensure that the data received is uncorrupted and that it is received in the correct order.
This is very important for the processing ond utilization of the data.
On the ground, we will need to interpret the data received from the rocket.
We need to be able to verify the rocket's height in order to acheive our goal.
We will also need to be able to find the rocket after it lands, which means using the data provided to us to locate it.
So long as these three main challenges are met, we will be able to complete our project.

\section{Proposed Solution}
The first challenge we will have to overcome is how to collect accurate data from the sensors.
Since the sensors will be reporting inaccurate data at high speeds and altitudes, we need a way to remove the noise from the readings.
Our proposed solution is to use a Kalman filter to filter out the noise.
A Kalman filter would allow us to predict sensor readings and adjust the predictions based on the actual readings.
It would also allow us to cross reference readings from all of the sensors and combine them into an internal representational model of the rocket.
This would provide us with more accurate readings, and a more accurate measure of success.
Additionally, we plan to attempt to lock GPS at apogee, which would give us a supremely accurate altitude reading.
With the more accurate data we will also be able to predict the landing location of the rocket, which will help us with recovery.
This is all dependent on the proper transmission of the data, which is why we plan on creating a UDP-like protocol to send the data to the ground.
In thinner atmosphere and at speeds over Mach 1, especially with potential interference from the body of the rocket itself, there is potential for high packet loss.
Because of this, a protocol which acknowledges the receipt of packets may not be viable.
Additionally, keeping track of the receipt of packets wastes precious processing power on the embedded system on the rocket.
This is why a protocol that assumes packets will be lost and can compensate for the missing packets is necessary.
Since we can check the data integrity on the receiving end and drop corrupt packets or interpolate missing ones, we don't have to worry about a few corrupt or missing packets.
Despite this ability, we still want to minimize packet loss while also minimizing processor use, so redundant transmission and reception devices would be preferred.
Once we have the data on the ground, we need to use it to track the rocket.
Tracking the rocket will involve modelling its trajectory on the ground using the data received.
This way we will be able to predict the landing location of the rocket.
Finally, we will try to capture and transmit GPS data after landing to aid in the location of the rocket.

\section{Performance Metrics}
The conditions for success for this project are that the rocket correctly reads and transmits data from the sensors, and that the rocket can be recovered.
We will be able to judge the performance of our Kalman filter based on how close the calculated altitude is to the one given by GPS.
This is of course dependent on being able to lock GPS at apogee.
We will be able to judge the performance of our data transmission on a couple of measures, the first being whether we can get live updating velocity and altitude data.
If we can get live updates from the rocket as it reaches landmarks in its flight, our protocol works as intended.
The second of these measures is whether the ground station has to do too many corrections or interpolations.
A few are expected, but if we have more than 50\% packet loss or corruption, we can say that our protocol is ineffective.
Our ground station's performance can be judged based on whether we are able to find the rocket as well as how accurate our predicted landing location is.
The accuracy of the predicted landing location is of course dependent on the performance of our Kalman filter and our communication protocol.
Therefore, we can also use the accuracy of the predicted landing location to judge their performance as well.

\end{document}
