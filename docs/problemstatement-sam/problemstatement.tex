\documentclass[onecolumn, draftclsnofoot,10pt, compsoc]{IEEEtran}
\usepackage{graphicx}
\usepackage{url}
\usepackage{setspace}

\usepackage{geometry}
\geometry{textheight=9.5in, textwidth=7in}

% 1. Fill in these details
\def \CapstoneTeamName{			CS Rocket Team}
\def \CapstoneTeamNumber{		42}
\def \GroupMemberOne{			Samuel Hudson}
\def \GroupMemberTwo{			Glenn Upthagrove}
\def \GroupMemberThree{			Michael Elliott}
\def \CapstoneProjectName{		100K Spaceport America Demonstration Rocket Project}
\def \CapstoneSponsorCompany{	Mechanical Engineering, Oregon State University}
\def \CapstoneSponsorPerson{		Nancy Squires}

% 2. Uncomment the appropriate line below so that the document type works
\def \DocType{		Problem Statement
				%Requirements Document
				%Technology Review
				%Design Document
				%Progress Report
				}
			
\newcommand{\NameSigPair}[1]{\par
\makebox[2.75in][r]{#1} \hfil 	\makebox[3.25in]{\makebox[2.25in]{\hrulefill} \hfill		\makebox[.75in]{\hrulefill}}
\par\vspace{-12pt} \textit{\tiny\noindent
\makebox[2.75in]{} \hfil		\makebox[3.25in]{\makebox[2.25in][r]{Signature} \hfill	\makebox[.75in][r]{Date}}}}
% 3. If the document is not to be signed, uncomment the RENEWcommand below
%\renewcommand{\NameSigPair}[1]{#1}

%%%%%%%%%%%%%%%%%%%%%%%%%%%%%%%%%%%%%%%
\begin{document}
\begin{titlepage}
    \pagenumbering{gobble}
    \begin{singlespace}
    	\includegraphics[height=4cm]{coe_v_spot1}
        \hfill 
        % 4. If you have a logo, use this includegraphics command to put it on the coversheet.
        %\includegraphics[height=4cm]{CompanyLogo}   
        \par\vspace{.2in}
        \centering
        \scshape{
            \huge CS Capstone \DocType \par
            {\large\today}\par
            \vspace{.5in}
            \textbf{\Huge\CapstoneProjectName}\par
            \vfill
            {\large Prepared for}\par
            \Huge \CapstoneSponsorCompany\par
            \vspace{5pt}
            {\Large\NameSigPair{\CapstoneSponsorPerson}\par}
            {\large Prepared by }\par
            Group\CapstoneTeamNumber\par
            % 5. comment out the line below this one if you do not wish to name your team
            \CapstoneTeamName\par 
            \vspace{5pt}
            {\Large
                \NameSigPair{\GroupMemberOne}\par
                \NameSigPair{\GroupMemberTwo}\par
                \NameSigPair{\GroupMemberThree}\par
            }
            \vspace{20pt}
        }
        \begin{abstract}
        % 6. Fill in your abstract    
        This document aims to outline problems our team will face and solutions we implement when launching a rocket into the mesosphere. Our objective as a CS team is to successfully track the rocket during its launch and descent. We will aim to design a UDP like protocol which will allow the rocket to communicate with our ground station on the earth’s surface. Our ground station will then send collected data to a web server where content will be served to a web application for accessible reports and diagnostics. We will be working closely with the electrical engineering team to design and integrate communication systems on the rocket. To ensure we meet our set requirements we will also create a series of unit tests for our software.
        \end{abstract}     
    \end{singlespace}
\end{titlepage}
\newpage
\pagenumbering{arabic}
\tableofcontents
% 7. uncomment this (if applicable). Consider adding a page break.
%\listoffigures
%\listoftables
\clearpage

% 8. now you write!
\section{Definition of Problem}
The 100k Spaceport America Demonstration Rocket Project presents a few interesting problems. Generally our objectives as a CS sub-team on this project are to successfully and accurately track an unguided rocket during its descent into the mesosphere and it’s return journey back to earth's surface. In addition there’s also a requirement to model and normalize informatics gathered from the rocket. We will need to correctly identify when an attempt of information sent from the rocket to our ground station receiver has been corrupted or lost in transmission. Information that is inaccurate or corrupted could jeopardize our ability to effectively track the rocket's flight path. Ensuring a good channel of unidirectional transmission from the rocket to our ground station is our primary object for this project as a cs team. Types of information we would expected to be able to receive include GPS coordinates, altitude data from an altimeter and accelerometer data. The rocket will be travelling at high speeds and will likely endure turbulence. This presents an additional problem as turbulence can affect telemetry. 
\section{Proposed Solution}
As a team we will look to design and implement a UDP like protocol to transmit data from a transmitter on the rocket to our ground station on earth’s surface. We will implement a simple parity bit error detection mechanism to help determine whether data has been corrupted or altered in transit. In addition, we will design a system to interpret, model and broadcast data gathered from the rocket. The system for interpreting the data will also be responsible for normalizing it. We will implement a filter to identify and rule out any outliers that exist. We will use historic data to render a flight trajectory to get an estimation on where the rocket will be heading. We want this information to be as accessible as possible so once the data is collected and verified we will convert this data into a common structured data language such as json and send this to a web application which will generate reports and graphs. The web application will also have an authentication mechanism which will allow for a remote rocket launch. The remote launch feature will have multi-factor authentication and will be highly secure. One great feature that we’re looking to add include for this project is live video stream from the rocket. People who have access to the web application will be able to view every stage of the launch from takeoff to recovery. 
\section{Performance Metric}
Successfully tracking and recovering the rocket after descent will be a big indication that our software works. There’s a few key indicators that will help us identify whether our tracking is accurate. For instance whilst the rocket is in-flight we will be gather GPS data to build a path of the rockets location. If this path is skewed fairly abruptly we can make the assumption that our normalization filter isn’t working correctly. If our data is rejected by the interpreter interface we can also assume that corrupt data is being accepted and our error detection mechanism isn’t working correctly. Another goal for this project is to build a web application that broadcasts this information for users interested in viewing the launch to see. If data from the ground station isn’t correctly packaged and sent to the application that will be another indication of failure to fulfil our goals as a cs team. We want to ensure we have a good testing environment for all of the software we design to support to successful launch of this rocket. 
\end{document}
